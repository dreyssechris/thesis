\chapter{Webanalyse} %ca 8-10 Seiten
\label{ch:webanalyse} 

\section{Definition und Abgrenzung} % 1-1,5 Seiten
Die Webanalyse (engl. Web Analytics) ist ein Teilbereich der Digitalanalyse (engl. Digital Analytics) und wird von der Digital Analytics Association (DAA) wie folgt definiert: \textit{"Web analytics is the measurement, collection, analysis and reporting of web data for purposes of understanding and optimizing web usage."} (DAA, 2008).

Eine weitere Definition liefert die ISO 19731:2017, welche Webanalyse als Analyse und Berichterstattung über das Verhalten, Aussagen und Stimmungen von Nutzern auf Online-Plattformen beschreibt [International Organization for Standardization (ISO), 2017, Abschnitt 3.93]. Diese Definition hebt zusätzlich die Bedeutung von qualitativen und stimmungsbasierten Daten hervor.

Die Abgrenzung zwischen Webanalyse und Digitalanalyse zeigt, dass sich die Webanalyse primär auf Daten bezieht, die aus dem Besuch und der Nutzung von Websites entstehen. Sie dient unter anderem dazu, die Herkunft von Nutzern (Kanäle), deren Klickverhalten (Klickpfade) sowie die Effektivität von Kampagnen und die Leistung einer Website zu analysieren. Erkenntnisse aus der Webanalyse unterstützen Unternehmen oder Organisationen dabei, ihre Inhalte entsprechend des Nutzerverhaltens zu personalisieren, Schwachstellen in der Customer Journey zu identifizieren und die Nutzererfahrung (User Experience) zu verbessern.
\section{Ziele} %1,5 Seiten
Die Durchführung von Webanalysen auf Bildungsportalen verfolgt das übergeordnete Ziel, die Plattformen kontinuierlich zu verbessern und eine nachhaltige sowie effektive Lernerfahrung zu ermöglichen. Im Folgenden werden die zentralen Ziele zusammengefasst.
\begin{enumerate}
    \item \textbf{Verbesserung der Lernprozesse:}
    Durch die Analyse von Webdaten können Bildungsportale herausfinden, welche Inhalte von Nutzern effektiv genutzt werden und welche weniger relevant sind. Dies ermöglicht eine gezielte Optimierung der Lerninhalte und eine Anpassung an die Bedürfnisse der Lernenden (Piwik PRO, o. J.).
    \item \textbf{Steigerung der Benutzerfreundlichkeit:}
    Die Untersuchung von Nutzerinteraktionen erlaubt es, die Benutzeroberfläche so zu gestalten, dass sie einfacher zu bedienen ist und den Lernprozess unterstützt. Dies trägt zu einer besseren Lernerfahrung bei und erleichtert die Navigation auf der Plattform (Piwik PRO, o. J.).
    \item \textbf{Personalisierung des Lernens:}
    Daten aus der Webanalyse ermöglichen es, individuelle Lernpfade zu entwickeln, die an die Fähigkeiten und Bedürfnisse einzelner Nutzer angepasst sind. Dies sorgt für eine effektivere und nachhaltigere Lernerfahrung (StudySmarter, o. J.).
    \item \textbf{Identifikation von Schwachstellen:} Die Analyse der Interaktionen und Nutzungsdaten kann Schwachstellen in der Customer Journey aufdecken, die anschließend gezielt beseitigt werden können, um die Lernerfahrung weiter zu optimieren (Piwik PRO, o. J.).
\end{enumerate}
Um die genannten Ziele der Webanalyse zu erreichen, ist es essenziell, die richtigen Kennzahlen zu identifizieren und zu verfolgen. Diese sogenannten Key Performance Indicators (KPIs) spielen eine zentrale Rolle, da sie Einblicke in das Nutzerverhalten und die Effektivität des Bildungsportals ermöglichen. Die relevanten KPI's für das Bildungsportal evaschiffmann.de, welche zum Erreichen dieser Ziele beitragen sollen werden im nächsten Kapitel vorgestellt.

\section{Key Performance Indicators} % 1-2

\section{Methoden der Datenerfassung} % 2-2,5 Seiten

\section{Tools zur Webanalyse} %3 Seiten























\section{Key Performance Indicators}
\begin{itemize}
\item Erklärung von KPI's im Kontext der Webanalyse
\item Relevante KPI's für Bildungswebsites (siehe Bounce Rate, Conversion Rate, Engagement, Custom Events)
\item Erkenntnisse aus verwandten Arbeiten und Studien.
\item Daraus abgeleitet und zusammen mit Betreibern der Website definierte KPI's für evaschiffmann.de
\item Wie werden verschiedene KPI's am besten auf dem Dashboard dargestellt (Zweck, Nutzergruppen beachten)
\end{itemize}
\subsection{Definition und Bedeutung}
\subsection{Relevante KPIs für Bildungsportale}
\subsection{Ableitung von KPIs für die Zielgruppe}












\section{Dashboardvisualisierung}
\begin{itemize}
    \item Prinzipien und Techniken der Datenvisualisierung
    \item Herausforderungen und Best Practices bei der Visualisierung von Daten.
    \item Anforderungen an Dashboards
    \item Vergleich ziehen zu verwandten Arbeiten
\end{itemize}

\subsection{Definition von Dashboards}
\subsection{Best Practices für Datenvisualisierunge}
\subsection{Anforderungen an Dashboards}
\subsection{Typen von Visualisierungen}