\chapter{Evaluation der implementierten Lösung}
\label{ch:evaluation}

\section{Methodik}
Definition der praktischen Bewertungskriterien:
\begin{itemize}
    \item Performance: Ladezeiten der Tracking-Skripte, Verarbeitungszeit der Events, Ladezeit des Dashboards
    \item Systemauslastung: CPU- und RAM-Nutzung auf Server- und Client-Seite
    \item Benutzerfreundlichkeit: Bedienbarkeit der Tools für Administratoren und Endnutzer
    \item (?)Feedback aus der Zielgruppe: Direkte qualitative Einschätzung der implementierten Lösung durch potenzielle Endnutzer (z. B. Lehrkräfte, Administratoren)
\end{itemize}

\section{Durchführung}
Performance- und Systemmessungen:
\begin{itemize}
    \item Beschreibung des Vorgehens zur Messung der Ladezeiten und Systemauslastung.
    \item Durchführung der Tests in der Live-Umgebung und Dokumentation der Ergebnisse
\end{itemize}
(?)Feedback aus der Zielgruppe:
\begin{itemize}
    \item Durchführung einer Feedback-Runde mit einer repräsentativen Gruppe der Zielnutzer
    \item Erstellen eines Fragebogens mit offenen und geschlossenen Fragen zur Benutzerfreundlichkeit, Nützlichkeit und Verständlichkeit des Dashboards
    \item Auswertung des Feedbacks durch qualitative Inhaltsanalyse
\end{itemize}

\section{Ergebnisse}
\begin{itemize}
    \item Darstellung der Ergebnisse der Performance-Messungen und des Feedbacks
    \item Identifikation von Stärken und Schwächen der implementierten Lösung
\end{itemize}

\section{Vergleich mit Anforderungen}
\begin{itemize}
    \item Vergleich der Ergebnisse mit den in Kapitel 2 definierten Anforderungen
    \item Bewertung, ob die implementierte Lösung die gestellten Anforderungen erfüllt und Verbesserungspotenziale bestehen
\end{itemize}