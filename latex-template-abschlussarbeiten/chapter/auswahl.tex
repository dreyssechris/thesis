% zusammen mit Umsetzung und Bewertung ca. 50 %
\chapter{Dashboardvisualisierung} 
\label{ch:auswahl}

\begin{itemize}
    \item Untersuchung, wie andere Forscher und Entwickler an ähnliche Aufgabenstellungen herangegangen sind
    \item Idee: Herausfinden, welche Tools für bestehende Bildungswebsites verwendet werden (in Grafik visualisieren)
\end{itemize}

\section{Kriterien für die Tools}

\subsection{Kriterien für Webanalyse-Tools}
Für die Bewertung der Webanalyse-Tools werden folgende Kriterien herangezogen (vorab):
\begin{itemize}
    \item \textbf{DSGVO-Konformität}: Die Einhaltung der Datenschutz-Grundverordnung ist eine essenzielle Voraussetzung für Bildungswebsites, da häufig personenbezogene Daten verarbeitet werden.
    \item \textbf{Kosten}: Die Tools sollten kosteneffizient sein und idealerweise Open-Source-Lizenzen besitzen, um langfristige Nutzung zu ermöglichen.
    \item \textbf{Flexibilität der Datenerfassung}: Die Möglichkeit, benutzerdefinierte Events zu erfassen und verschiedene Arten von Nutzerdaten zu tracken, ist entscheidend.
    \item \textbf{Skalierbarkeit}: Die Tools sollten in der Lage sein, auch bei steigendem Nutzeraufkommen zuverlässig zu funktionieren.
    \item \textbf{Integration in bestehende Systeme}: Die Kompatibilität mit der bestehenden Infrastruktur (z. B. TYPO3) und API-Schnittstellen zur Weiterverarbeitung der Daten ist wichtig.
    \item \textbf{Export der erfassten Daten}: Eine einfache Möglichkeit zum Exportieren der erfassten Daten in verschiedenen Formaten ist von Vorteil.
    \item \textbf{Dokumentation}: Die Qualität der Dokumentation beeinflusst die Implementierungszeit und Wartbarkeit des Tools erheblich.
\end{itemize}

\subsection{Kriterien für Dashboardvisualisierungstools}
Für die Bewertung der Dashboardvisualisierungstools werden folgende Kriterien definiert:
\begin{itemize}
    \item \textbf{Visualisierungsoptionen}: Die Tools sollten eine breite Palette an Diagrammtypen und Layouts bieten, um die Daten effektiv darzustellen.
    \item \textbf{Interaktivität}: Die Möglichkeit, Filter und Drill-Down-Funktionen zu nutzen, um tiefere Einblicke zu erhalten, ist essenziell.
    \item \textbf{Benutzerfreundlichkeit}: Ein intuitives Interface ist wichtig, damit auch nicht-technische Benutzer die Dashboards bedienen können.
    \item \textbf{Kosten und Lizenzierung}: Die Tools sollten möglichst kostengünstig sein und idealerweise unter einer Open-Source-Lizenz stehen.
    \item \textbf{Kompatibilität}: Die Tools müssen nahtlos mit dem ausgewählten Webanalyse-Tool und der bestehenden Infrastruktur zusammenarbeiten.
    \item \textbf{Dashboard-Funktionalität}: Das Tool sollte die Möglichkeit bieten, benutzerdefinierte Dashboards zu erstellen und diese individuell anzupassen.
\end{itemize}

\subsection{Begründung und Gewichtung der Kriterien}
(*unfertig)Die oben genannten Kriterien wurden ausgewählt, um sicherzustellen, dass die eingesetzten Tools den Anforderungen einer Bildungswebsite gerecht werden. Dabei spielt insbesondere die DSGVO-Konformität eine zentrale Rolle, da Bildungswebsites häufig von minderjährigen Nutzern besucht werden. Die Gewichtung der Kriterien erfolgt auf Basis ihrer Relevanz für die Website evaschiffmann.de. So wird der DSGVO-Konformität und der Benutzerfreundlichkeit ein besonders hohes Gewicht beigemessen, da diese Faktoren entscheidend für den langfristigen Betrieb und die Akzeptanz der Lösung sind.

\section{Vergleich der Tools}

\subsection{Vergleich der Webanalyse-Tools}

\subsection{Vergleich der Dashboardvisualisierungstools}
Die Dashboardvisualisierungstools werden anhand der festgelegten Kriterien verglichen. Der Fokus liegt dabei auf der Interaktivität, der Benutzerfreundlichkeit und der Möglichkeit zur Erstellung benutzerdefinierter Dashboards.

\section{Ergebnisse des Vergleichs}
\begin{itemize}
    \item Bildung eines Rankings basierend auf der Bewertung der Tools
    \item Bestimmung des theoretisch am besten geeigneten Webanalyse-Tools und Dashboardvisualisierungstools basierend auf den Kriterien und der Bewertung
\end{itemize}

Die Ergebnisse des Vergleichs bilden die Grundlage für die praktische Implementierung der ausgewählten Tools in Kapitel \ref{ch:implementierung}.
