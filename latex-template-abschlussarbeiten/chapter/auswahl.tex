\chapter{Dashboardvisualisierung} % 5-6 Seiten? 
\label{ch:auswahl}
Im vorherigen Kapitel wurde beschrieben, wie Nutzerdaten erfasst werden können. Hierfür wurden relevante KPIs für das Bildungsportal definiert und verschiedene Methoden der Datenerhebung sowie deren DSGVO-konforme Umsetzung beschrieben. Um die gesammelten Daten strukturiert und übersichtlich darzustellen und der Professur für Geschichtsdidaktik eine fundierte Grundlage für datenbasierte Erkenntnisse zu bieten, sollen die KPIs visuell auf einem Dashboard aufbereitet werden. Dazu werden zunächst die Anforderungen an das Dashboard erörtert. Anschließend werden geeignete Visualisierungsmethoden für die KPIs vorgestellt und die Struktur sowie der Aufbau des Dashboards festgelegt. Abschließend wird Grafana als Visualisierungslösung evaluiert. Hierbei wird auf das Kapitel~\ref{sec:anforderungen} Bezug genommen und beschrieben wie technisch-funktionale-,sowie Usability- und Design Anforderungen in Grafana realisiert werden können. 

\section{Anforderungen an das Dashboard}
\label{sec:anforderungen}
\subsection{Technisch-funktionale Anforderungen}
Die technisch-funktionalen Anforderungen definieren, welche Funktionen ein Dashboard bereitstellen muss und welche technologischen Rahmenbedingungen erfüllt sein sollten, um eine effiziente und stabile Nutzung zu gewährleisten. Diese Anforderungen sind: 
\begin{itemize}
    \item \textbf{Datenintegration und Aggregation:} Das Dashboard Tool soll in der Lage sein, Daten aus Matomo oder alternativ direkt aus der Datenbank zu beziehen und zu aggregieren. Außerdem sollen die per Webanalysetool erfassten Daten automatisch aktualisiert werden. Entweder in definierten Intervallen oder in Echtzeit.
    \item \textbf{Datenintegration und Aggregation:}
\end{itemize}

\subsection{Usability- und Design Anforderungen}

\section{Visualisierungsmethoden}
Neben den Funktionalen und Technischen Anforderungen an das Dashboard Lösung, sollten diese ebenfalls  

\section{Struktur und Aufbau des Dashboards}
Ein Dashboard hat die Aufgabe komplexe Daten verständlich und übersichtlich darzustellen. Durch die visuelle Aufbereitung können Muster und Zusammenhänge leichter erkannt werden. Eine klare und intuitive Benutzeroberfläche reduzieren nicht nur die kognitive Belastung, sondern erleichtern ebenfalls die Informationssuche. Hierzu ist es wichtig relevante Daten zu priorisieren. Um dies zu erreichen wird eine progressive Drill-Down-Struktur angestrebt. Eine solche Struktur sorgt dafür, dass allgemeinere KPIs, wie Seitenaufrufe oder die Absprungrate weiter oben im Dashboard angezeigt werden.
\section{Grafana als Visualisierungslösung}
Für Vergleich der Datenanbindungsmöglichkeiten: https://alexandre.deverteuil.net/post/visualize-matomo-metrics-in-grafana/












