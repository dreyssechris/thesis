\chapter{Einleitung}
\label{ch:einleitung}

\section{Motivation}
\label{sec:motivation}

Die Nutzung von Online-Bildungsportalen hat sich in den letzten Jahren zu einem zentralen Bestandteil moderner Bildungsstrategien entwickelt und nimmt weiterhin stetig zu. Laut Eurostat nutzten im Jahr 2023 etwa 30 \% der Internetnutzer in der Europäischen Union im Alter von 16 bis 74 Jahren Online-Kurse oder digitale Lernmaterialien – ein Anstieg von 2 Prozentpunkten im Vergleich zum Vorjahr \parencite{Eurostat}. Diese Entwicklung unterstreicht nicht nur die wachsende Bedeutung digitaler Bildungsangebote, sondern auch deren transformative Wirkung auf die Wissensvermittlung. Aus einer Untersuchung der Potomac University geht des Weiteren hervor, dass 70 \% der Universitätsstudenten Online-Lernen als vorteilhafter im Vergleich zum traditionellen Unterricht bewerten \parencite{Potomac}. Die Ergebnisse verdeutlichen, wie digitale Lernplattformen die Art der Wissensvermittlung nachhaltig beeinflussen und eine sinnvolle Ergänzung zu traditionellen Lehrmethoden darstellen.

Angesichts dieser Entwicklung stellt sich vor allem aus didaktischer, aber auch aus technischer Sicht die Frage, wie das Lernverhalten auf Bildungsportalen analysiert werden kann. Für die Professur für Geschichtsdidaktik der Friedrich-Schiller-Universität Jena, welche das Bildungsportal \textit{evaschiffmann.de} betreibt, gewinnt diese Fragestellung zunehmend an Bedeutung. Es ist von besonderem Interesse zu verstehen, wie Nutzer mit dem digitalisiertem historischen Wissen umgehen. Vor diesem Hintergrund soll eine Lösung geschaffen werden, welche die Professur bei der Untersuchung des Nutzerverhaltens unterstützt.

\section{Zielsetzung}
\label{sec:zielsetzung}

In einem gemeinsamen Gespräch mit Herrn Staack von der Professur für Geschichtsdidaktik wurden die Anforderungen an die Datenanalyse definiert. Dabei wurde besonderes Augenmerk auf die detaillierte Nachvollziehbarkeit einzelner Nutzersitzungen gelegt. Es soll möglich sein, zu analysieren, welche Seiten ein Nutzer aufruft und wie dieser mit den Elemente der Website interagiert. Besonders relevant sind die Interaktionen mit den Kernelementen des Bildungsportals – den Tagebucheinträgen und den interaktiven Überschriften. Zudem soll das Tool Informationen darüber liefern, aus welchen Quellen Besucher auf die Website gelangen. Darüber hinaus ist es von Interesse, wie viele Besucher das Bildungsportal nutzen, wie lange sie verweilen und welche Bereiche der Website bevorzugt werden.

Bestehende Webanalysetools bieten umfassende Funktionen zur Datenerfassung, sind jedoch beschränkt in den Visualisierungsmöglichkeiten der Analysewerte. Für spezifischere Visualisierungen fallen zu dem meist zusätzliche Kosten an. Zudem sind die relevanten Informationen meist nicht kompakt zusammengefasst, sondern über verschiedene Unterseiten verstreut, was die Übersichtlichkeit verringert und eine gezielte Analyse erschweren kann.

Darüber hinaus sind existierende Lösungen nicht explizit auf die Anforderungen der Nutzeranalyse auf dem Bildungsportal zugeschnitten. Insbesondere fehlt eine geeignete Lösung zur individuellen Analyse einer Nutzersitzung, die sowohl Interaktionen mit spezifischen Website-Elementen als auch Seitenwechsel mit den jeweiligen Zeitstempeln erfasst, um die Dauer einzelner Aktionen präzise zu bestimmen und diese ebenfalls gut aufbereitet und ansprechend visualisiert.

Ziel dieser Arbeit ist es daher, ein geeignetes Webanalysetool sowie ein kompatibles Dashboardtool für die Visualisierung der Informationen zu identifizieren und zu implementieren. Die Lösung soll die erfassten Daten strukturiert aufbereiten und eine datenschutzkonforme Analyse des Nutzerverhaltens auf dem Bildungsportal ermöglichen.

\section{Forschungsfrage}
\label{sec:forschungsfrage}
Aus den genannten Zielen und Herausforderungen ergibt sich folgende Forschungsfrage für die Bachelorarbeit:

\textit{„Wie kann ein Web-basiertes Analysetool für das Bildungsportal \textit{evaschiffmann.de} entwickelt werden, das eine datenschutzkonforme Erhebung und eine effektive Visualisierung von Nutzerdaten für eine aussagekräftige Analyse ermöglicht?“}

\section{Aufbau der Arbeit}
\label{sec:aufbau}
Die Arbeit besteht aus sieben Kapiteln. Den Anfang bildet die Einleitung, gefolgt vom theoretischen Teil in Kapitel zwei und drei. In diesem theoretischen Teil werden grundlegende Begriffe erläutert. Explizite Analysewerte zu den Anforderungen werden definiert und die Funktionsweise der Webanalyse wird in Kapitel zwei beschrieben. Kapitel drei befasst sich mit dem Dashboard und den Visualisierungsmöglichkeiten. In Kapitel 4 wird das Konzept für die Implementierung, sowie zur Beantwortung der Forschungsfrage geschildert. Die Implementierung wird in darauffolgenden Kapitel beschrieben und im sechsten Kapitel wird diese anschließend evaluiert. Abschließend werden in Kapitel sieben die erzielten Ergebnisse bewertet und zusammengefasst.