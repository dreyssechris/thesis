\chapter{Einleitung}
\label{ch:einleitung}

\section{Motivation}
\label{sec:motivation}

Die Nutzung von Online-Bildungsportalen hat sich in den letzten Jahren zu einem zentralen Bestandteil moderner Bildungsstrategien entwickelt und nimmt weiterhin stetig zu. Laut Eurostat nutzten im Jahr 2023 etwa 30 \% der Internetnutzer in der Europäischen Union im Alter von 16 bis 74 Jahren Online-Kurse oder digitale Lernmaterialien
in den drei Monaten vor Durchführung der Studie – ein Anstieg von 2 Prozentpunkten im Vergleich zum Vorjahr [Eurostat, 2024]. Diese Entwicklung unterstreicht nicht nur die wachsende Bedeutung digitaler Bildungsangebote, sondern auch deren transformative Wirkung auf die Wissensvermittlung. Aus einer Untersuchung der Potomac University geht des Weiteren hervor, dass 70 \% der Universitätsstudenten Online-Lernen als vorteilhafter im Vergleich zum traditionellen Unterricht bewerten [Potomac University, 2024]. Die Ergebnisse verdeutlichen, wie digitale Lernplattformen die Art der Wissensvermittlung nachhaltig beeinflussen und eine sinnvolle Ergänzung zu traditionellen Lehrmethoden darstellen.

Angesichts dieser Entwicklung stellt sich vor allem aus didaktischer, aber auch aus technischer Sicht die Frage, wie das Lernverhalten auf Bildungsportalen analysiert werden kann. Für die Professur für Geschichtsdidaktik der Friedrich-Schiller-Universität Jena, welche das Bildungsportal \textit{evaschiffmann.de} betreibt, gewinnt diese Fragestellung zunehmend an Bedeutung. Es ist von besonderem Interesse zu verstehen, wie Nutzer mit dem digitalisiertem historischen Wissen umgehen. Vor diesem Hintergrund soll eine Lösung geschaffen werden, welche die Professur für Geschichtsdidaktik bei der Untersuchung des Nutzerverhaltens unterstützt.

\section{Zielsetzung}
\label{sec:zielsetzung}

(*überarbeiten, am Ende)Das Ziel dieser Bachelorarbeit ist es, ein geeignetes Webanalysetool sowie ein passendes Dashboardvisualisierungstool für das Bildungsportal zu identifizieren und zu implementieren. Das Webanalysetool soll ermöglichen, zentrale Key Performance Indicators (KPIs) wie Nutzerzahlen, Abbruchrate, Verweildauer, Scrolltiefe und benutzerdefinierte Events datenschutzkonform zu erfassen. Zudem soll das Visualisierungstool die gewonnenen Daten so aufbereiten, dass die relevanten KPIs verständlich und übersichtlich dargestellt werden können. Das Hauptaugenmerk liegt dabei auf der Wahl eines geeigneten Layouts sowie der Verwendung sinnvoller Diagrammtypen, um die Daten benutzerfreundlich und klar zu präsentieren.
Zur Erreichung dieses Ziels...

\section{Aufbau der Arbeit}
\label{sec:aufbau}
(*überarbeiten, am Ende)Die Arbeit besteht aus sieben Kapiteln. Den Anfang bildet die Einleitung, gefolgt vom theoretischen Teil in Kapitel zwei und drei. In diesem theoretischen Teil werden grundlegende Begriffe definiert sowie der aktuelle Stand der Forschung und Technik erläutert. Kapitel zwei widmet sich dem Thema Webanalyse und den Key Performance Indicators, während Kapitel drei das Thema Dashboardvisualisierung behandelt.

Kapitel vier beinhaltet eine Analyse verschiedener Tools hinsichtlich ihrer Eignung zur Erfassung und Visualisierung von Daten. Auf Grundlage der Ergebnisse dieser Analyse wird im fünften Kapitel der ausgewählte Toolstack in einer Testumgebung implementiert. Im sechsten Kapitel folgt die Entwicklung und Anwendung einer Evaluationsstrategie zur Bewertung der Lösung. Abschließend werden in Kapitel sieben die erzielten Ergebnisse bewertet und zusammengefasst.