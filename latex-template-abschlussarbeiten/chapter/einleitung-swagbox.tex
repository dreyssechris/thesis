\chapter{Einleitung}
\label{ch:einleitung}

\section{Motivation}
\label{sec:motivation}

(*unfertig)Die Nutzung von Online-Bildungsportalen hat sich in den letzten Jahren zu einem zentralen Bestandteil moderner Bildungsstrategien entwickelt und nimmt weiterhin an Bedeutung zu. Laut Eurostat nutzten im Jahr 2023 etwa 30 \% der Internetnutzer in der Europäischen Union im Alter von 16 bis 74 Jahren Online-Kurse oder digitale Lernmaterialien in den drei Monaten vor Durchführung der Studie – ein Anstieg von 2 Prozentpunkten im Vergleich zum Vorjahr [Eurostat, 2024]. Diese Entwicklung unterstreicht nicht nur die wachsende Bedeutung digitaler Bildungsangebote, sondern auch deren transformative Wirkung auf die Wissensvermittlung. Ergänzend zeigt eine Untersuchung der Potomac University, dass 70 \% der Universitätsstudenten Online-Lernen als vorteilhafter im Vergleich zum traditionellen Unterricht bewerten (Potomac University, 2023). Diese Wahrnehmung verdeutlicht das Potential digitaler Lernplattformen, die Bildung nachhaltig zu beeinflussen und traditionelle Lehrmethoden zu ergänzen.

Angesichts dieser Entwicklung stellt sich vor allem aus didaktischer, aber auch aus technischer Sicht die Frage, wie das Lernverhalten auf Bildungsportalen analysiert werden kann. Für die Professur für Geschichtsdidaktik der Friedrich-Schiller-Universität Jena, welche das Bildungsportal \textit{evaschiffmann.de} betreibt, gewinnt diese Fragestellung zunehmend an Bedeutung. Es ist von besonderem Interesse, zu verstehen, wie Nutzer mit dem digitalisiertem historischen Wissen umgehen und wie das Nutzerverhalten auf der Plattform effektiv nachvollzogen und visualisiert werden kann.

Vor diesem Hintergrund gewinnt die Auswahl eines geeigneten Analysetools
immer mehr an Bedeutung. Es existieren zahlreiche Lösungen, welche
Einblicke in Nutzungsdaten ermöglichen und dadurch zum besseren Verständnis 
über die Nutzerinteraktionen auf solchen Plattformen beitragen können. 
Eine fundierte Bewertung anhand definierter Kriterien der am besten geeigneten Werkzeuge 
für diesen Anwendungsfall kann dazu beitragen tiefgreifende Einsichten über das Verhalten auf 
Bildungsportalen zu erlangen.

*Damit leistet die Auseinandersetzung mit diesem Thema einen wertvollen Beitrag
zur Verknüpfung moderner technischer Analysemethoden mit didaktischer Forschung
und fördert ein tieferes Verständnis für den Umgang mit digitalen Bildungsportalen.

\section{Zielsetzung}
\label{sec:zielsetzung}

(*unfertig)Ziel dieser Bachelorarbeit ist es, verschiedene Tools zur Webanalyse und Datenvisualisierung zu implementieren und zu vergleichen, um das jeweils am besten geeignete Tool für die Datenerfassung und die Dashboard-Visualisierung für die Website evaschiffmann.de zu identifizieren und auf dem tatsächlichen Server der Plattform zu implementieren.

Die Tools werden dabei nicht nur hinsichtlich ihrer Funktionalität und Benutzerfreundlichkeit bewertet, sondern auch hinsichtlich ihrer Integration in die bestehende technische Infrastruktur und ihrer Fähigkeit, für die Zielgruppe relevante Daten verständlich und übersichtlich aufzubereiten.

\section{Metodik}
\label{sec:metodik}

\section{Aufbau der Arbeit}
\label{sec:aufbau}


