%! suppress = UnresolvedReference


\chapter{Persönliche Reflexion, Fazit und Ausblick}
\label{ch:zusammenfassung}

\section{Persönliche Reflexion}
Die Umsetzung dieser Bachelorarbeit war sowohl fachlich als auch organisatorisch herausfordernd. Da das ursprünglich vorgesehene TYPO3-Projekt, das als Grundlage für die Analyse des Bildungsportals dienen sollte, nicht rechtzeitig zur Verfügung stand, konnte der geplante Meilensteinplan nicht wie vorgesehen eingehalten werden. Um diese Situation zu überbrücken, wurde zunächst mit den theoretischen Kapiteln begonnen und eine konzeptionelle Ausarbeitung der Lösung vorgenommen. Gleichzeitig wurden die Tools zunächst in eine im Laufe des Studiums erstellte Website integriert, um trotzdem Erfahrungen für die Implementierung und Konfiguration zu sammeln. Eine weitere Herausforderung war die vollständige Einrichtung der Container-Umgebung. Ebenso bestanden zu Beginn der Arbeit keine Vorkenntnisse in der Konfiguration und dem Aufsetzen von Webservern, weshalb sich dieses Wissen im Verlauf der Arbeit zusätzlich angeeignet werden musste. Anfangs waren die Dienste lediglich über Angabe separate Ports für die Dienste in der URL erreichbar. Erst durch weitere Recherche konnte eine Lösung mit einem Reverse Proxy umgesetzt werden. Über diesen konnten alle Dienste unter einer gemeinsamen Domain und individuellen Pfaden zugänglich gemacht werden, wodurch die Integration in die Live-Umgebung später erleichtert wird. Auch die Integration der Matomo-Daten in Grafana stellte zunächst eine Herausforderung dar. Die ursprüngliche Idee, hierfür die Reporting-API von Matomo zu verwenden, erwies sich als zu unflexibel, da die benötigten Informationen nicht in dem Detail gefiltert werden konnten, wie es für die Analysewerte erforderlich war. Insbesondere im Bezug auf die selbstdefinierten Grafana-Variablen. Um die Metriken und KPIs in der gewünschten Detailtiefe und Aktualität zu erfassen, war es erforderlich, spezifische SQL-Abfragen auf die Datenbank von Matomo anzuwenden. 

Trotz oder auch gerade wegen diesen Hürden wurde im Verlauf der Arbeit sehr viel neues Wissen erlangt und bestehendes Wissen vertieft. Es wurde keine vorgefertigte Lösung übernommen, sondern auf Grundlage eigener Anforderungen und intensiver Recherche eine individuelle Lösung für das Bildungsportal entwickelt. Dabei konnte nicht nur fundiertes Wissen im Bereich der Webanalyse und datenschutzkonformen Datenverarbeitung aufgebaut werden, sondern auch die praktische Umsetzung eines Dashboards realisiert werden. Zudem wurden bestehende Kenntnisse in SQL, JavaScript, Python und Docker-Compose erweitert und gefestigt.

Insgesamt konnte ich durch diese Arbeit wertvolle Erfahrungen in der Softwareentwicklung und insbesondere im Bereich der Webanalyse und Datenvisualisierung sammeln – sowohl technisch als auch konzeptionell. 

\section{Fazit}
Ziel dieser Arbeit war die Entwicklung eines Web-basierten Analysetools für das Bildungsportal \textit{evaschiffmann.de}, das eine datenschutzkonforme Erhebung sowie eine effektive und aussagekräftige Visualisierung von Nutzerdaten ermöglicht. Im Rahmen der Implementierung wurde ein vollständiges Tracking- und Visualisierungssystem konzipiert und implementiert, das auf den Open-Source-Tools Matomo und Grafana basiert.

Die im Konzept definierten Anforderungen an Datenschutz, Datenintegration, Aggregation sowie Sicherheit und Zugriffskontrolle, welche in den Kapiteln~\ref{ch:webanalyse} und~\ref{ch:auswahl} motiviert wurden, konnten erfolgreich umgesetzt werden. Dies wurde im Rahmen der technischen Evaluation aufgezeigt. Auch die Nutzerevaluation durch Herrn Staack zeigt, dass die entwickelte Lösung den gewünschten Anforderungen in hohem Maße gerecht wird. Die grafischen Darstellungen, Filterfunktionen und Visualisierungen wurden insgesamt als hilfreich, verständlich und benutzerfreundlich empfunden. Besonders die Analyse einzelner Nutzersitzungen im Gantt-Diagramm und die Sinnhaftigkeit der ausgewählten Analysewerte wurde hierbei sehr positiv bewertet. Die Verbesserungspotenziale, also die Ergänzung eines Events zur Erfassung von Bildinteraktionen und die Möglichkeit zur separaten Auswertung einzelner Tagebuchseiten, konnten aus zeitlichen Gründen nicht mehr vollständig umgesetzt werden. Diese Aspekte wurden in der Evaluation klar benannt und können als Grundlage für eine mögliche letzte Optimierung verwendet werden.

Die Arbeit zeigt insgesamt, dass es möglich ist, kostenlos und mit frei verfügbaren Tools eine datenschutzkonforme und dennoch aussagekräftige Webanalyse zu realisieren. Die entwickelte Lösung kann im Live-Betrieb dazu beitragen, besser zu verstehen, wie Nutzer mit dem digitalisierten historischen Wissen auf dem Bildungsportal umgehen. Darüber hinaus kann die Lösung dafür verwendet werden, eventuelle Verbesserungspotentiale der Website zu identifizieren.

Die zentrale Forschungsfrage für die Arbeit lautete: 

\textit{„Wie kann ein Web-basiertes Analysetool für das Bildungsportal \textit{evaschiffmann.de} entwickelt werden, das eine datenschutzkonforme Erhebung und eine effektive Visualisierung von Nutzerdaten für eine aussagekräftige Analyse ermöglicht?“}

Die durchgeführte Implementierung und die Evaluation zeigen auf, wie eine solche Lösung konkret realisiert werden kann, wodurch die Forschungsfrage beantwortet ist. Die entwickelte Anwendung erfüllt die datenschutzrechtlichen Vorgaben, ermöglicht eine strukturierte Analyse der erhobenen Daten und unterstützt durch die grafische Aufbereitung im Dashboard effektiv die Auswertung des Nutzerverhaltens. Damit wurde das Ziel dieser Arbeit erreicht.

\section{Ausblick}
Die entwickelte Lösung zur Analyse des Nutzerverhaltens ist nicht ausschließlich auf das Bildungsportal \textit{evaschiffmann.de} beschränkt, sondern lässt sich grundsätzlich auch auf andere Bildungsportale und Websites übertragen. Insbesondere die im Dashboard enthaltene Darstellung einzelner Nutzersitzungen in Form eines Gantt-Diagramms bietet eine flexible und visuell anschauliche Möglichkeit, Nutzerinteraktionen zeitlich geordnet abzubilden. Durch eine entsprechende Anpassung der erfassten Events kann dieses Diagramm auch Seitenwechsel, Interaktionen innerhalb von Seiten sowie das Verlassen der Seite über externe Links auf anderen Webseiten abbilden.

Für zukünftige Arbeiten ergeben sich verschiedene Forschungsansätze. Denkbar wäre etwa, alternative Formen zur Visualisierung einzelner Nutzersitzungen zu erproben, um komplexe Interaktionsverläufe noch anschaulicher darzustellen. Ebenso könnten Verfahren zur automatisierten Erkennung typischer Nutzungspfade mithilfe KI-gestützter Analysemodelle entwickelt werden. Solche Verfahren könnten dabei helfen, das Verhalten auf Bildungsplattformen nicht nur schneller, sondern auch präziser auszuwerten. Gerade im Kontext digitaler Lernangebote eröffnen sich dadurch neue Wege, um das Nutzerverhalten besser zu verstehen. 



