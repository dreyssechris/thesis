\pdfbookmark[0]{Kurzfassung}{MKurzfassung}
\begin{abstract}
    \section*{Kurzfassung}
    \label{sec:Kurzfassung}

    Diese Abschlussarbeit, welche im Rahmen eines Bachelorstudiums an der Fachhochschule Erfurt im Studiengang Angewandte Informatik angefertigt wurde, befasst sich mit dem Design und der Implementierung eines Web-basierten Analysetools zur Untersuchung des Nutzerverhaltens auf einem Bildungsportal. Ziel der Arbeit ist die Entwicklung einer datenschutzkonformen Lösung zur Erfassung, Auswertung und Visualisierung von Nutzungsdaten am Beispiel des Bildungsportals \textit{evaschiffmann.de}. 

    Im Mittelpunkt steht die Integration des Webanalyse-Tools Matomo sowie die Darstellung und Aufbereitung der erfassten Daten über das Dashboard-Tool Grafana. Dazu wurden geeignte Metriken und Key Performance Indicators definiert, welche eine gezielte Analyse der Nutzerinteraktionen ermöglichen. Ein wesentlicher Bestandteil der Analyse ist die Nachvollziehbarkeit einzelner Nutzersitzungen sowie der Auswertung von Interaktionen mit den zentralen Elementen der Website wie Tagebucheinträgen und Überschriften. 
    
    Die Implementierung erfolgte in einer containerisierten Umgebung mithilfe von Docker und Docker Compose, um eine einfache Reproduzierbarkeit und eine problemlose Bereitstellung im Live-Betrieb zu gewährleisten. Die entwickelte Anwendung wurde sowohl technisch als auch mittels einer Nutzerbefragung in Form eines Fragebogens evaluiert. Die Ergebnisse der Evaluation zeigen, dass die Lösung eine praktikable und übersichtliche Möglichkeit zur Analyse des Nutzerverhaltens darstellt.

    Die Arbeit richtet sich in erster Linie an die Anwender der Professur für Geschichtsdidaktik, welche die entwickelte Lösung im Rahmen ihrer Forschung nutzen werden. Darüber hinaus wendet sie sich an Bildungseinrichtungen und Privatpersonen, welche an der Erhebung und Visualisierung von Nutzerdaten im Bildungsbereich interessiert sind.

\end{abstract}

\pdfbookmark[0]{Abstract}{MAbstract}
\begin{abstract}
    \section*{Abstract}
    \label{sec:Abstract}
    This bachelor thesis, developed as part of a degree in Applied Computer Science at the University of Applied Sciences Erfurt, focuses on the design and implementation of a web-based analytics tool for examining user behavior on an educational platform. The aim of the thesis is to develop a solution for collecting, analyzing, and visualizing usage data for the educational website \textit{evaschiffmann.de}.

    At the core of the project is the integration of the web analytics tool Matomo, as well as the presentation and processing of the collected data using the dashboard tool Grafana. To support targeted analysis of user interactions, appropriate metrics and key performance indicators were defined. A key component of the analysis is the ability to track individual user sessions and evaluate interactions with central elements of the website, such as diary entries and collapsible headings.

    The implementation takes place within a containerized environment using Docker and Docker Compose to ensure reproducibility and facilitate future deployment in a live setting. The developed application was evaluated both technically and through a user survey in the form of a questionnaire. The evaluation confirmed that the tool proves to be a practical and effective solution for the structured analysis of user behavior.

    The thesis is primarily intended for the users at the Chair of History Didactics at the Friedrich Schiller University Jena, who will use the developed analytics tool in their research. Furthermore, it is aimed at educational institutions and private individuals interested in collecting and visualizing user data in the field of digital education.

\end{abstract}
