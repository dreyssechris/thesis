\chapter{Einleitung}
\label{ch:einleitung}

\section{Motivation}
\label{sec:motivation}

Die Nutzung von Online-Bildungsportalen hat sich in den letzten Jahren zu einem zentralen Bestandteil moderner Bildungsstrategien entwickelt und nimmt weiterhin stetig zu. Laut Eurostat nutzten im Jahr 2023 etwa 30 \% der Internetnutzer in der Europäischen Union im Alter von 16 bis 74 Jahren Online-Kurse oder digitale Lernmaterialien – ein Anstieg von 2 Prozentpunkten im Vergleich zum Vorjahr \parencite{Eurostat}. Aus einer Untersuchung der Potomac University geht des Weiteren hervor, dass 70 \% der Universitätsstudenten Online-Lernen als vorteilhafter im Vergleich zum traditionellen Unterricht bewerten \parencite{Potomac}. Diese Umfragen verdeutlichen, dass digitale Bildungsportale die Wissensvermittlung nachhaltig verändern und mittlerweile traditionelle Lehrmethoden ergänzen. Sie zeigen auf, wie wichtig digitale Angebote inzwischen geworden sind und welches Potenzial diese für die Wissensvermittlung haben.

Angesichts dieser Entwicklung stellt sich vor allem aus didaktischer, aber auch aus technischer Sicht die Frage, wie das Lernverhalten auf Bildungsportalen analysiert werden kann. Für die Professur für Geschichtsdidaktik der Friedrich-Schiller-Universität Jena, welche das Bildungsportal \textit{evaschiffmann.de} betreibt, gewinnt diese Fragestellung zunehmend an Bedeutung. Es ist von besonderem Interesse zu verstehen, wie Nutzer mit dem digitalisierten, historischen Wissen umgehen. Vor diesem Hintergrund soll eine Lösung geschaffen werden, welche die Professur bei der Untersuchung des Nutzerverhaltens unterstützt.

\section{Zielsetzung}
\label{sec:zielsetzung}

In einem gemeinsamen Gespräch mit Herrn Staack von der Professur für Geschichtsdidaktik wurden die Anforderungen an die Datenanalyse definiert. Dabei wurde besonderes Augenmerk auf die detaillierte Nachvollziehbarkeit einzelner Nutzersitzungen gelegt. Es soll möglich sein, zu analysieren, welche Seiten ein Nutzer aufruft und wie dieser mit den Elementen der Website interagiert. Besonders relevant sind die Interaktionen mit den Kernelementen des Bildungsportals – den Tagebucheinträgen und den interaktiven Überschriften. Zudem soll die Lösung Informationen darüber liefern, aus welchen Quellen Besucher auf die Website gelangen. Darüber hinaus ist es von Interesse, wie viele Besucher das Bildungsportal nutzen, wie lange sie verweilen und welche Bereiche der Website bevorzugt werden.

Bestehende Webanalyse-Tools bieten umfassende Funktionen zur Datenerfassung, sind jedoch beschränkt in den Visualisierungsmöglichkeiten der Analysewerte. Für spezifischere Visualisierungen fallen zu dem meist zusätzliche Kosten an. Des Weiteren sind die relevanten Informationen meist nicht kompakt zusammengefasst, sondern über verschiedene Unterseiten verstreut, was die Übersichtlichkeit verringert und eine gezielte Analyse erschweren kann.

Darüber hinaus sind existierende Lösungen nicht explizit auf die Anforderungen der Nutzeranalyse auf dem Bildungsportal zugeschnitten. Insbesondere fehlt eine geeignete Lösung zur individuellen Analyse einer Nutzersitzung, welche sowohl Interaktionen mit bestimmten Website-Elementen als auch Seitenwechsel mitsamt Zeitstempeln erfasst, um daraus die Dauer einzelner Aktionen präzise abzuleiten und diese ansprechend zu visualisieren.

Ziel dieser Arbeit ist es daher, ein dediziertes Tool für die Datenerfassung sowie ein kompatibles Tool für die Visualisierung der Informationen zu identifizieren und zu implementieren. Durch den Einsatz eines separaten Visualisierungstools ergeben sich erweiterte Möglichkeiten zur individuellen Konfiguration und Darstellung der erfassten Daten. Die entwickelte Lösung soll die Nutzungsdaten strukturiert aufbereiten und eine datenschutzkonforme Analyse des Nutzerverhaltens auf dem Bildungsportal ermöglichen.

\section{Forschungsfrage}
\label{sec:forschungsfrage}
Aus den genannten Zielen und Herausforderungen ergibt sich folgende Forschungsfrage für die Bachelorarbeit:

\textit{„Wie kann ein Web-basiertes Analysetool für das Bildungsportal \textit{evaschiffmann.de} entwickelt werden, das eine datenschutzkonforme Erhebung und eine effektive Visualisierung von Nutzerdaten für eine aussagekräftige Analyse ermöglicht?“}

\section{Aufbau der Arbeit}
\label{sec:aufbau}
Die Arbeit gliedert sich in sieben Kapitel. Nach der Einleitung folgt der theoretische Teil in Kapitel~\ref{ch:webanalyse} und~\ref{ch:auswahl}. Kapitel~\ref{ch:webanalyse} behandelt die Erhebung der Analysedaten und stellt das verwendete Webanalyse-Tool vor. Zudem werden entsprechend der Anforderungen an die Analyse gezielte Analysewerte definiert. Kapitel~\ref{ch:auswahl} widmet sich der Visualisierung der erfassten Daten und stellt das hierfür eingesetzte Visualisierungstool vor. In Kapitel~\ref{ch:konzept} wird das Konzept zur Implementierung sowie zur Beantwortung der Forschungsfrage erläutert. Die praktische Umsetzung wird in Kapitel~\ref{ch:implementierung} beschrieben, bevor in Kapitel~\ref{ch:evaluation} eine Evaluation der entwickelten Lösung erfolgt. Kapitel~\ref{ch:zusammenfassung} bildet den Abschluss und fasst die zentralen Ergebnisse zusammen, die im Hinblick auf die Zielsetzung der Arbeit bewertet werden. In diesem Kapitel wird ebenfalls die Forschungsfrage beantwortet.
