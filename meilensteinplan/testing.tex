\documentclass{article}

\usepackage[utf8]{inputenc}
\usepackage[ngerman]{babel}
\usepackage[T1]{fontenc}
\usepackage[table]{xcolor}
\usepackage[margin=1in]{geometry}
\usepackage{longtable}
\usepackage{enumitem}

\setlist{nolistsep}
\definecolor{myGreen}{HTML}{006666}

\renewcommand{\familydefault}{\sfdefault}
\renewcommand{\arraystretch}{1.5}

\title{\textbf{Meilensteinplan Bachelorarbeit}}
\author{Chris Dreyße}
\date{Oktober 2024}

\begin{document}
\maketitle

\begin{center}
    \begin{longtable}{>{\bfseries}p{0.25\textwidth} p{0.6\textwidth} p{0.15\textwidth}}
    \arrayrulecolor{myGreen}\hline
    \textcolor{myGreen}{Phase} & \textcolor{myGreen}{Aufgaben} & \textcolor{myGreen}{Dauer} \\ \hline
    \endfirsthead

    \hline
    \textcolor{myGreen}{Phase} & \textcolor{myGreen}{Aufgaben} & \textcolor{myGreen}{Dauer} \\ \hline
    \endhead

    Vorbereitungsphase & & Oktober/November \\ 
    1. Orientierung/Planung & 
    \begin{itemize}
        \item[1.1] Themenfindung und Eingrenzung
        \item[1.2] Erste Literaturrecherche zu Telemetrie und Usability-Optimierung
        \item[1.3] Prüfen der Realisierbarkeit
    \end{itemize} & \\ 
    
    2. Methodenauswahl und Konzeptentwicklung & 
    \begin{itemize}
        \item[2.1] Rücksprache mit Baxxler zu geeigneter Nutzeranalyse/Visualisierung
        \item[2.2] Rücksprache mit Betreuer
        \item[2.3] Eingrenzen geeigneter Open-Source Tools
        \item[2.4] Erste Einarbeitung in Tools und Techniken, Dokumentationen lesen
    \end{itemize} & \\ \hline
    
    Recherche- und Strukturierungsphase \textcolor{myGreen}{(Beginn BA Zeitraum - 30.11.)} & & 2,5 Wochen \\ 
    3. Recherche, Materialbeschaffung & 
    \begin{itemize}
        \item[3.1] Intensivierte Literaturrecherche für theoretischen Hintergrund
        \item[3.2] Quellensammlung erstellen
        \item[3.3] Grobe Gliederung bis in die zweite Ebene erstellen
        \item[3.4] Grobes Konzept für den praktischen Teil erstellen
    \end{itemize} & \\ \hline
    
    Schreib- und Implementierungsphase & & 5 Wochen \\ 
    4. Theoretischer Teil (Grundlegend) & 
    \begin{itemize}
        \item[4.1] Format und Layout erstellen
        \item[4.2] Einleitung und theoretische Grundlagen schreiben
        \item[4.3] Gliederung verfeinern
    \end{itemize} & \\ 
    
    5. Implementierung & 
    \begin{itemize}
        \item[5.1] Endgültige Festlegung der Tools für Telemetriedatensammlung/Visualisierung
        \item[5.2] Methodik für die Datenerfassung festlegen
        \item[5.3] Konzeption des Dashboards
        \item[5.4] Datenerhebung und Sammlung der Nutzerverhaltensdaten
        \item[5.5] Implementierung des Dashboards
    \end{itemize} & \\ \hline
    
    Schreib- und Korrekturphase & & 2,5 Wochen \\ 
    6. Dokumentation des Praktischen Teils & 
    \begin{itemize}
        \item[6.1] Daten auswerten und Verbesserungspotenziale identifizieren
        \item[6.2] Technische Beschreibung der Implementierung
        \item[6.3] Analyse und Auswertung der Ergebnisse
        \item[6.4] Feedback vom Betreuer einholen und integrieren
    \end{itemize} & \\ 
    
    7. Theoretischer Teil (Verknüpfen von Theorie und Praxis) & 
    \begin{itemize}
        \item[7.1] Überarbeitung und Vervollständigung der Einleitung und theoretischer Abschnitte
        \item[7.2] Erweiterung des theoretischen Teils mit praktischen Erkenntnissen
        \item[7.3] Bearbeiten noch offener/unvollständiger Gliederungspunkte
        \item[7.4] Schlussfolgerungen und Empfehlungen
    \end{itemize} & \\ \hline
    
    Pufferzeit & Zusätzliche Woche für unvorhergesehene Verzögerungen & 1 Woche \\ \hline
    
    Abgabephase & & 1 Woche \\ 
    8. Abschluss & 
    \begin{itemize}
        \item[8.1] Anpassung der Arbeit an formale Vorgaben (Deckblatt, Layout)
        \item[8.2] Korrekturlesen
        \item[8.3] Finales Feedback vom Betreuer einholen und integrieren
        \item[8.4] Drucken und Binden der Arbeit
        \item[8.5] Abgabe
    \end{itemize} & \\ \hline
    
    Gesamtdauer & & 12 Wochen \\ \hline
    \end{longtable}
\end{center}

\end{document}