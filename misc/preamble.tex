%%--------------------------------------
%%   Packages
%%--------------------------------------

\usepackage[utf8]{inputenc}        	% Zeichenkodierung - !!!Achtung, alle Dateien auch im UTF8 speichern!!!
\usepackage[T1]{fontenc}            % westeuropäische Schriftzeichenkodierung
\usepackage[ngerman]{babel}        	% Babel-System - neue deutsche Rechtschreibung
\usepackage[left=4cm, right=3cm, top=2.5cm, bottom=2.5cm]{geometry} % Seitenränder



% Erlaubt sinnvolle URL-Umbrüche
\usepackage[hyphens]{url}
\usepackage{breakurl}
\usepackage{etoolbox}



% Literaturverzeichnis kompakter (aber mit etwas Abstand)
\apptocmd{\bibsetup}{\sloppy\clubpenalty4000\widowpenalty4000\relax}{}{}

% Optional: Etwas mehr Stretch für lange URLs
\setlength{\emergencystretch}{1.5em}


%---------------------------------------
%  Befehl für Schriftart Helvet / Arial
%---------------------------------------
\usepackage{helvet}						% Alternative Schriftart: lmodern
\renewcommand{\familydefault}{\sfdefault}	%serifenlos

\usepackage[onehalfspacing]{setspace}   % 1,5 facher Zeilenabastand aber nur im text nicht in Fußnoten oder Verzeichnissen

\usepackage{amssymb}        			% Mathesymbole
\usepackage{amsfonts}        			% mathematische Schriftarten
\usepackage{amsmath}       				% Mathepaket
\usepackage{cancel}            			% Durchstreichungen wie beim kürzen
\usepackage{mathcomp}        			% weitere Symbole
\usepackage{scrhack}        			% verbessert einige Fremdklassen in Zusammenspiel mit KomaScript
\usepackage[babel,german=quotes]{csquotes}	%Zitate
\usepackage[ngerman]{translator}		% Übersetzt Wörter die Latex setzt korrekt (wie z.B. "`Abbildung"')
%\usepackage{epstopdf}					
\usepackage{tikz}           		 	% Für selbsterstellte Graphiken
\usepackage{booktabs}        			% Für schönere Tabellen
\usepackage{array}                  	% erweiterte Tabelleneigenschaften
\usepackage{graphicx}               	% Grafiken
\usepackage{subfigure}              	% Grafiken nebeneinander mit (a) und (b)


\usepackage[hyphens]{url} % erlaubt Umbrüche in URLs
\usepackage{breakurl} 



%--------------------------------------
%   PDF Lesezeichen und Hyperlinks
%--------------------------------------
\usepackage[
pdfauthor={\autor},
pdftitle={{\titel { - }\untertitel}},
pdfsubject={{\titel { - }\untertitel}},
pdfkeywords={\keywords},
pdfpagelabels = {true},
pdfstartview = {FitV},
colorlinks = {true},
linkcolor = {black},
citecolor = {black},
urlcolor = {blue},
bookmarksopen = {true},
bookmarksopenlevel = {3},
bookmarksnumbered = {true},
plainpages = {false},
hypertexnames = {false}
]{hyperref}


%%--------------------------------------
%%   Glossar
%%--------------------------------------
%\usepackage[toc,acronym,nonumberlist]{glossaries}
%\makeglossaries




%--------------------------------------
%   Kopf- & Fußzeile 
%--------------------------------------
\usepackage[automark,plainheadsepline,autooneside]{scrlayer-scrpage}
\pagestyle{scrheadings}
\clearpairofpagestyles                                  % Kopf und Fußzeile löschen
\ihead[\hochschule]{\hochschule}                        % im Kopf -> links
\ohead[\fachgebiet]{\fachgebiet}                        % im Kopf -> rechts
\cfoot[\pagemark]{\pagemark}                            % Seitenzahl -cfoot für center - ofoot für outer
\renewcommand*{\headfont}{\upshape\sffamily\scriptsize} % Schrift Kopfzeile
\renewcommand*{\footfont}{\normalfont\sffamily\small}   % Schrift Fußzeile


%--------------------------------------
%   Quellcode-Listing Einstellungen
%   ftp://ftp.tu-chemnitz.de/pub/tex/macros/latex/contrib/listings/listings.pdf
%--------------------------------------

\usepackage{xcolor,listings}               	% bindet das Paket Listings ein
\definecolor{comment}{rgb}{.15,.4,.15}     	% hellgruen
\definecolor{keywd1}{rgb}{.15,.15,.6}      	% dunkelblau
\definecolor{keywd2}{rgb}{.35,.5,.55}      	% hellblau
\definecolor{string}{rgb}{.5,.15,.15}      	% dunkelrot
\definecolor{gray}{rgb}{0.4,0.4,0.4}		
\definecolor{darkblue}{rgb}{0.0,0.0,0.6}
\definecolor{cyan}{rgb}{0.0,0.6,0.6}


% Hier kann jetzt alles für die verwendete Sprache eingestellt werden
% Der lstset-Befehl ermöglicht haufenweise Einstellungen zur Formatierung

\lstset{language=C++,                            % hier Sprache einstellen
    basicstyle={\small} ,                        % Schriftgröße
    keywordstyle=\color{blue!80!black!100},      % Farbe der keywords
    identifierstyle=,                            % Bezeichnerstyle, hier leer
    commentstyle=\color{green!50!black!100},     % Farbe der Kommentare
    stringstyle=\ttfamily,                       % Aussehen der Strings
    breaklines=true,                             % Automatische Zeilenumbrüche
    numbers=left,                                % Zeilennummerierung links
    numberstyle=\small,                          % Größe der Zeilennummerierung
    frame=single,                                % einfacher Rahmen
    backgroundcolor=\color{blue!3},              % Hintergrundfarbe
    caption={Code},                              % Standardüberschrift
    captionpos=t,                                % Überschift oben (top)
    showspaces=false,							 % Leerzeichen nicht anzeigen
    showstringspaces=false,						 % Stringleerzeichen nicht anzeigen
% %UTF8 gebastle verdammtes tex grml
    literate= %
        {Ä}{{\"A}}1
        {Ö}{{\"O}}1
        {Ü}{{\"U}}1
        {ß}{{\ss}}1
        {ä}{{\"a}}1
        {ö}{{\"o}}1
        {ü}{{\"u}}1
        {~}{{\textasciitilde}}1
}

%\lstdefinelanguage{XML}
%{
%morestring=[b]",
%morestring=[s]{>}{<},
%morecomment=[s]{<?}{?>},
%stringstyle=\color{black},
%identifierstyle=\color{darkblue},
%keywordstyle=\color{cyan},
%morekeywords={xmlns,version,type}% list your attributes here
%}



%----------Worttrennung-----------
%---Latex trennt eigentlich recht gut, aber Fremdwörter o.ä. manchmal nicht, daher kann man Latex das Trennen einzelner Wörter beibringen, z.B.:
%\hyphenation{Ein-gangs-pro-zess}
%\hyphenation{Aus-gangs-pro-zess}
%\hyphenation{Web-client}
%\hyphenation{DMS-Web-client}
